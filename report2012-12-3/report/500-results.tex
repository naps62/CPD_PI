\section{Numerical Results}
\label{sec:500}
We tested the two methods in the context of the one-dimnesional geometry where we deal with two representative situations:
the transport of a very smooth function  and a discontinuous function. We focus the analyse on two particular points:
the accuracy of the methods for smooth solution and the stability of the methods for rough solution.
We assume that the velocity is $u=1$ and prescribe periodic boundary condition. The mesh is a uniform $I+1$ points
subdivision with space parameter $\Delta x=\frac{1}{I}$. Time step $\Delta t$ is controled by the CFL condition and we 
shall set $\Delta t=0.6 \Delta x$.
In the sequel, $S1$ is the first-order scheme while $S2MUCL$ and $S2MOOM$ are the second-order scheme with the MUSCL and 
the MOOD limiting strategies respectively.
\subsection{The smooth case}
The initial solution is the simple sine function $\phi^0(x)=\sin(2\pi x)$ we transport at velocity $u=1$.
We plot in figure (\ref{fig_sine}) the exact solution and the approximations using he the MUSCL and the MOOD method
after a complete revolution. Clearly, the MOOD method provides the best approximation and manage to reduce
the diffusiove effect of the FV scheme. We report in figure (\ref{fig_conv1}-\ref{fig_convinfty}) the $L^1$ and $L^\infty$
convergence curve and we get an effective second-order convergence. We remark that the MOOD technique gives a smaller
error but the convergence rates are indentical in the two cases.
We do not provide the table convergence data since the curves are really straighforward. 
\begin{figure}[ht]
\begin{center}
\begin{tabular}{c}
\includegraphics[width=7cm,height=5cm, clip=true,viewport=80 200 600 580
]{figure_sin.pdf}
\end{tabular}
\end{center}
\caption{\label{fig_sine} \footnotesize Numerical solutions at $t=1.0$ s for the sine function: 
exact solution in red, muscl method in green, mood method in blue.}
\end{figure}


\begin{figure}[t]
\begin{center}
\begin{tabular}{c}
\includegraphics[width=7cm,height=4cm, clip=true,viewport=80 200 600 580
]{curve1.pdf}
\end{tabular}
\end{center}
\caption{\label{fig_conv1}  \footnotesize $L^1$ error convergen using the log-log scale: muscl method in green, mood method in blue.
We observe an effective second-order convergence.}
\end{figure}

\begin{figure}[t]
\begin{center}
\begin{tabular}{c}
\includegraphics[width=7cm,height=4cm, clip=true,viewport=80 200 600 580
]{curve0.pdf}
\end{tabular}
\end{center}
\caption{\label{fig_convinfty}  \footnotesize $L^\infty$ error convergen using the log-log scale: 
muscl method in green, mood method in blue.
We observe an effective second-order convergence.}
\end{figure}

\subsection{The rough case}
The initial solution is the Heaviside function centered at $1/2$ given by $\phi^0(x)=H(x-1/2)$ 
we transport at velocity $u=1$.
We plot in figure (\ref{fig_heavi}) the exact solution and the approximations using he the MUSCL and the MOOD method
after a complete revolution. One more time, the MOOD method provides the best approximation and manage to reduce
the diffusiove effect of the FV scheme in the vicinity of the discontinuity.
We do not provide any convergence curve since we deal with a discontinuous function. Nevertheless, we have check a 
half-order convergence in norm $L^1$ which characterize the convergence rate for a discontinuous solution.
\begin{figure}[ht]
\begin{center}
\begin{tabular}{c}
\includegraphics[width=7cm,height=5cm, clip=true,viewport=80 200 600 580
]{figure_heavi.pdf}
\end{tabular}
\end{center}
\caption{\label{fig_heavi} Numerical solutions at $t=1.0$ s for the step function: 
exact solution in red, muscl method in green, mood method in blue.}
\end{figure}




