\subsection{MUSCL method}
\label{sec:410}
The MUSCL method is based on a local limitation of the slope to provide stability.
Actually, maximum principle is the main criterion and write
\begin{equation}
\min_{j\in \nu(i)}(\phi^n_i,\phi_j)\leq u^{n+1}_i \leq \max_{j\in \nu(i)}(\phi^n_i,\phi_j).
\label{eq:MP_criterion}
\end{equation}
To achieve such a property, we multiply the reconstructed slope $\widehat \sigma_i$ by a
coefficient $\chi_i\in[0,1]$ such that with the new slope $\sigma_i=\chi_i\widehat \sigma_i$
the update solution $\phi^{n+1}_i$ satisfies the restriction (\ref{eq:MP_criterion}).\\
There exists a lot of method to determine the limiter but we propose the one introduce
by Barth and Jepherson. To this end, let denote 
$$
\widehat \phi_{ij}=\phi_i+\widehat \sigma_i B_iM_{ij}
$$
where $M_{ij}$ is the midpoint of edge $e_{ij}$. In a first step, 
for any cell $K_i$ and $j\in\nu(i)$, we evaluate the quantity
$$
\chi_{ij} = \left\{\begin{array}{l l}
\frac{\max(\phi_j,\phi_i)-\phi_i}{\phi_{ij}-\phi_i} & \textrm{ if } \phi_{ij}> \max(\phi_j,\phi_i) \\
\frac{\min(\phi_j,\phi_i)-\phi_i}{\phi_{ij}-\phi_i}  &\textrm{ if } \phi_{ij} <\min(\phi_j,\phi_i) \\
1 & \text{otherwise}
\end{array}\right.
$$
Then we define 
$$
\chi_i=\min_{j\in\nu(i)} \chi_{ij}
$$
and the reconstructed values by
$$
u_{ij} = u_i + \chi_i\widehat \sigma_i  B_i M_{ij}
$$


For the one-dimensional situation, we use the limiter operator such as the minmod function
setting
$$
\chi_i=(\phi_{i+1}-\phi_{i})\text{minmod}\left (\frac{\phi_{i}-\phi_{i-1}}{\phi_{i+1}-\phi_{i}}\right ) 
$$
with minmod$(\alpha)=0$ if $\alpha<0$;  minmod$(\alpha)=1$ if $\alpha>1$ and minmod$(\alpha)=\alpha$
otherwise.
We then compute the reconstructed values with
$$
\phi_{i-1/2}^{n,+}=\phi_i^n-\chi_i^n \widehat \sigma_i^n \frac{\Delta x}{2},\quad
\phi_{i+1/2}^{n,-}=\phi_i^n+\chi_i^n \widehat \sigma_i^n \frac{\Delta x}{2}
$$
The algorithm for a one forward Euler step is
\begin{enumerate}
\item mean values $(\phi^n_i)$ are known.
\item compute the slope $\sigma_i$
\item compute the limiter $\chi_i$
\item compute the reconstructed values
\item compute the flux
\item time update 
\end{enumerate}
One has to perform two time the procedure when using the RK2 technique to provide
second-order in time.\\
{\it Remark.} The scheme is {\it a priori} because the limiting procedure has been performed before
the update.  
 