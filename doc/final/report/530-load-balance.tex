\subsection{Load Balance}
\label{subsec:mpi:load}

\todorev{Last revised on Sat, June 30 at 23:15 by pfac}

With the naive partitioning strategy used, load balance becomes a problem.
The computation itself is actually well balanced, since it is assured that every partition has the same amount of cells (differing at most by one). The problem is in the border between those partitions.
With the division by the horizontal coordinate being used, it becomes obvious that the size of the border between partitions becomes extremely dependant on the format of the mesh itself.
Other approaches, already mentioned in \cref{subsubsec:mpi:partitioning:research} attempt to deal with this, and produce partitions that minimize the size of the border.

The drawback of not controlling border sizes comes at the communication step.
Not only a different partitioning solution could minimize the border, thus minimizing the amount of data transfered, it can also happen that different partitions have very different border sizes, compromising communication balance.

\todonaps{Anhe? Esta última frase faz grande sentido para mim.}
