\section{Case Study}
\todo[inline]{Explain what the program is meant for}

The application analyzed in this document computes, called here \texttt{polu}, computes the flux of a material (e.g. a pollutant) through a bidimensional surface. This surface is described as a mesh, composed mainly of edges and cells. The input is given by a 

The original \texttt{polu} application works like a heartbeat algorithm with no communication (since it is executed in a single computation node). The algorithm used by the application sees the environment as a discrete mesh (represented by its cells and edges) and loops until the specified time interval is reached. At each iteration of this main loop, the algorithm performs two main steps:

\begin{description}[\IEEEsetlabelwidth{Pollution Update}\IEEEusemathlabelsep]
	\item[Flux Computation] Based on current pollution values for each cell, the flux for each edge is calculated (performed by the \texttt{compute\_flux} function)
	\item[Pollution Update] Using the previously calculated flux values, the pollution for each cell is updated (performed by the \texttt{update} function)
\end{description}

\subsection{Algorithm}


The algorithm used by the \texttt{polu} consists on a computation stage, where all the required elements are loaded and prepared, and two computation stages, composing the main loop.

Operations performed in the preparation stage are highly dependent on the implementation being described, as most will require some elements to be properly organized or some values to be previously computed. Common operations, such as loading the mesh, the initial polution state and the velocities vector are constant to every implementation, but they may still differ in the structures used to store the data.

A single execution of the two computation stages together form a step in the iterative method behind this application. These stages, also referred in this document as core functions, are the \texttt{compute\_flux} and \texttt{update} functions.

In \texttt{compute\_flux}, all the edges in the mesh are analyzed, and the flux of polution to be moved across that edge is computed, based on the polution level and the velocity vectors of the cells it connects. A preconfigured value is used as the Dirichlet condition, which replaces the polution level of a second cell for the edges in the border of the mesh.

As for the \texttt{update} function, it uses the computed flux values to update the polution levels of each cell in the mesh, by adding the individual contribution of each edge of the cell. While triangular cells are prefered, there are no restrictions to the number of edges a cell may have.

The access patterns of the core functions make this algorithm a typical case of a stencil computation. In terms of performance, this implies the number of operations per accessed byte will most likely remain constant with larger problem sizes.
\todo[inline,color=red!40]{Review this last part.}
\todo[inline,color=red!40]{Might be useful to explain how parallelism is made with common stencils, so their dreams can be crushed in the next subsection. Either here or at the beginning of the next subsection.}

\subsection{Parallelism Oportunities}
\todo[inline]{Explain what can be executed in parallel}
\todo[inline]{Mention this is a heartbeat}

During the main loop of the program both core functions, \computeflux and \update, depend on each other to perform their tasks. \update requires flux from all edges to be previously computed in \computeflux, which in turn requires that all pollution values are up-to-date to compute flux for the next iteration. This creates two implicit synchronization points in the main loop, and is a consequence of the heartbeat characteristics of the problem.

This allows both functions to be looked at as individual tasks, that may be subject to different parallelization approaches.
Both functions perform calculations using the entire mesh

