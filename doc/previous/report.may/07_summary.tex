\section{Summary}
\label{sec:summary}

\Cref{tab:summary} shows a summary of all implemented versions of the algorithm so far. It should be highlighted that both the Environmental Setup and the testing Methodology differ between some tests. For instance. the initial tests, with the \textit{Original}, \textit{OpenMP AOS} and \textit{CUDA} implementations were performed using the SeARCH Group 511, and a full execution of the program (which, for the largest test case used, does around 20,000 iterations of the main loop).

The last two versions, \textit{OpenMP SOA} and \textit{MPI}, gave the best results on SeARCH Group Hex, and were only tested for a fixed amount of iterations, due to issues of availability of the nodes, which required a smaller amount of time to be requested. This approach depends on the homogeneity of the main loop (the workload does not change between iterations) and allows valid tests to be performed requiring only a fraction of the total time. Yet, since the amount of iterations is much smaller than the full execution would be, the impact of the preprocessing is much higher, especially in the MPI implementation, where the mesh partitioning introduces an even larger preprocessing time, when compared to the other versions.

\begin{table}[!htp]
		\smaller
		\begin{center}
			\begin{tabular}{l r ll}
			\hline
			\textbf{Implementation} & \textbf{\~ Time (s)}  & \textbf{Node Group}	& \textbf{Execution}	\\ \hline
			Original				& 6200				& 511				& Full							\\
			OpenMP AOS				& 1900				& 511				& Full							\\
			CUDA 					& 200				& 511 				& Full							\\ \hline
			OpenMP SOA 				& 36				& Hex 				& 1000 iter.					\\
			MPI 					& 38				& Hex 				& 2000 iter.					\\
			\hline
			\end{tabular}
		\end{center}
		\caption{Summary of all the \texttt{polu} versions implemented so far. Note that the environmental setup and methodology changed as the project evolved.}
		\label{tab:summary}
	\end{table}

Specially when taking into account the differences in the methodology, the CUDA implementation appears to be the one with the best performance gain. It is possible that the \textit{OpenMP SOA} and the \textit{MPI} implementations actually provide better execution times for a full execution, if it is considered that an amount of the time shown is spent in preprocessing. However, \cref{tab:summary} shows only the values for the best results, with the ``ideal'' number of processes and/or nodes. When the amount of processing units is increased, the speedups were shown not to be linear (see \cref{fig:speedup}).