\section{Introduction}
\label{sec:100}

The work presented started with an implementation of a first order finite volume method applied to a bidimensional domain. The domain is represented as a 2D mesh, composed of cells connected by their interfaces, and contains a velocity field $\vec{V}(x, y)$ to control the direction of propagation. This velocity field is kept constant throughout the entire computation. For simplification purposes, it has only one direction:

$$V = \left( \begin{array}{c} 1 \\ 0 \end{array} \right) $$

The case study and application of this method is to compute the distribution of a pollutant in a surface, and its distribution as time passes.

The report is organized as follows. \Cref{sec:200} explains the first order method initially used. \Cref{sec:300} details how the linear reconstruction is performed on the second order methods, which are described in \Cref{sec:400}. Numerical and efficiency results are presented in \cref{sec:500} and finally \cref{sec:900} gathers some conclusions obtained from the work.