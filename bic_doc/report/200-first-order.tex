\section{First Order Scheme}
\label{sec:200}

The initial approach consisted on a first order scheme that solves the problem.

$$ \int_{K_i} \partial{t} u + dir(V_{n_{ij}})u = 0 $$

Let $K_i$ and $K_j$ be two neighbor cells of the system, and $e_{ij}$ the interface between them. In a first step, the method computes the flux passing from $K_i$ to $K_j$, based on the current average values for each cell, and the velocity field $\vec{V}$. The flux is computed for each edge using the immediate neighbor information. The first-order finite volume scheme used is as follows:

$$ u_i^{n+1} = u_i^n - \Delta{t} \sum_{j \in \underline{v}(i)} {|e_{ij}| \over |K_j|} F(u_i^n, u_j^n, n_{ij}) $$

Where the flux calculation for each interface is:

$$ F(u_i^n, u_j^n, n_{ij}) = [V_{n_{ij}}]^{+}u_i^n + [V_{n_{ij}}]^{-}u_j^n $$

The implemented method works on a bidimensional mesh based on some assumptions. One of the assumptions is that edges should always have at least on adjacent cell, and that cell will be refered to as the left cell. The other cell, called the right cell, may not exist, which is the case for the edges on the border of the domain. On those cases, it is assumed:

$$ u_j = dc $$

Where $dc$ is the \textit{Dirichlet Boundary Condition}. In this particular work, this value was kept constant throughout the computation.

First other schemes like these are very popular due to its simplicity, but suffer from large amounts of numerical diffusion, which leads to poor accuracy and the smoothing of discontinuities as time passes. However, this approach was useful to provide something to work on top of, when targeting the more accurate approaches with the second-order schemes.